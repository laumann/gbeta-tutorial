\epigraph{I understand my tests are popular reading in the teachers' lounge}{Calvin}

No tutorial can rightfully call itself so if it lacks the
quintessential \texttt{Hello, World!} program. Accordingly, here is
the very first \gbeta program, \texttt{hello.gb}:

\lstinputlisting[caption=hello.gb]{gbsrc/hello.gb}

We can already note a few things from this example:

\begin{itemize}\setlength{\itemsep}{-2pt}
  \item Strings in \gbeta are a basic type. Literal strings are
    enclosed in single quotes (').
  \item The first line: \verb|-- universe:descriptor --|\ will be
    treated as `magic' for now, and explained in the section
    concerning the fragment system.
  \item The text \texttt{Hello, World!} is \emph{assigned} to
    \texttt{stdio}. In \gbeta\ assignments are on the form:
    \texttt{value|var}. Think of it as the value on the left side
    being `piped' into the right side.\footnote{bash programmers can
      think of the piping mechanism. For programmers used to the
      \texttt{var x = value} syntax this corresponds to \texttt{value|x}.}
\end{itemize}